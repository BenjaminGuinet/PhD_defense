\chapter{Objectifs de thèse}
{\hypersetup{linkcolor=GREYDARK}\minitoc}
\label{chap:intro-objectifs}

\section{Le parasitoïdisme comme facteur explicatif de la variabilité d'endogénisation et domestication chez les Hyménoptères}

Aujourd'hui, les événements d'endogénisation virale commencent à être bien documentés \citep{ter_horst_endogenous_2019, cheng_nudivirus_2020,kondo_novel_2019,irwin_systematic_2022,li_hgt_2022,flynn_assessing_2019}. À l'échelle des insectes, le paysage des EVEs est très variable entre les espèces \citep{gilbert_diversity_2022}. Jusqu'à présent, la taille du génome ainsi que la qualité des assemblages étaient les seuls facteurs proposés pour expliquer cette variabilité. En plus de ces facteurs, nous pourrions également tester l'impact de facteurs écologiques. En effet, les Hyménoptères ayant des habitudes endoparasitaires semblent présenter une plus grande propension à endogéniser et domestiquer des entités virales comparées à des espèces au style de vie ectoparasitaires (pondant leurs œufs sur un hôte) ou des espèces vivant librement tels que les fourmis ou les abeilles. Cette hypothèse est motivée par un ensemble d'études de cas documentant les 6 domestications virales évoquées précédemment chez les guêpes endoparasitoïdes \citep{volkoff_analysis_2010,beliveau_genomic_2015-1,pichon_recurrent_2015,burke_rapid_2018,di_giovanni_behavior-manipulating_2020}. Ces 6 cas concernent exclusivement des endoparasitoïdes et des virus donneurs à génome à ADN double brin.

Face à ces observations, nous proposons à travers le (\hyperref[sec:chap1]{chapitre1}) d'aborder plusieurs questions. 

(1) La première partie de la thèse vise à tester si le phénomène d'endogénisation et de domestication virale est un phénomène plus largement répandu chez les Hyménoptères en élargissant notre point de vue et incluant un panel plus représentatif d'espèces réparties dans la diversité des Hyménoptères. (2) Ensuite, nous proposons de tester si la propension à endogéniser et domestiquer du matériel génétique viral est lié ou non aux virus donneurs puisque jusqu'à présent, ces phénomènes particuliers de domestication semblent impliquer uniquement des virus à grand génome à ADN double brins (dsDNA). (3) Enfin, puisque jusqu'à présent l'ensemble des cas de domestications virales ont été liés à des espèces au style de vie endoparasitoïdes, nous proposons de tester l'hypothèse selon laquelle le style de vie pourrait être un facteur explicatif de la variabilité d'endogénisation et de domestication chez les Hyménoptères. 
%Dans le cas d'un rejet de l'hypothèse nulle (cf. pas d'effet du style de vie), nous pourrions conclure 
Nous nous attendons à ce que les espèces endoparasitoïdes présentent une propension plus grande à endogéniser et/ou domestiquer du matériel génétique viral comparé à des espèces au style de vie libre ou ectoparasitoïde, et ceci pour plusieurs raisons : 

La première raison vient de l'intimité de l'interaction entre l'œuf ou la larve du parasitoïde et l'hôte. En d'autres termes, le mode de vie endoparasitaire pourrait faciliter l'acquisition de nouveaux virus provenant des hôtes du fait de cette proximité. De plus, le mode de vie endoparasitoïde pourrait également faciliter le maintien et la propagation des virus nouvellement acquis au sein des populations de guêpes. En effet, les guêpes endoparasitoïdes injectent souvent non seulement des œufs, mais aussi des composés venimeux (typiquement produits dans la glande à venin ou dans les cellules du calice) où des virus peuvent être présents et peuvent donc être transmis verticalement \citep{martinez_additional_2016, coffman_viral_2022, stasiak_characteristics_2005}.  En outre, le confinement de plusieurs guêpes en développement au sein d'un même hôte peut faciliter la transmission horizontale de ces virus (par exemple \cite{varaldi_infectious_2003}).

La deuxième raison est  liée à l'intensité de la pression de sélection exercée par le système immunitaire de l'hôte sur les endoparasitoïdes. Cette pression sélective peut favoriser la cooptation de fonctions virales telles que l'activité de fusion membranaire mentionnée ci-dessus, qui constitue un moyen très efficace pour délivrer des facteurs de virulence. 

Afin de tester ces hypothèses, j'ai développé un pipeline bioinformatique pour détecter les événements d'endogénisation impliquant tout type de virus (ADN/ARN, simple brin, double brin), à l'échelle de l'ensemble de l'ordre des Hyménoptères. Cette analyse nous a d'abord permis de tester si la propension des virus à entrer dans les génomes des Hyménoptères, et à être domestiqués, dépend de leur structure génomique.  Nous avons ensuite testé si le mode de vie de l'espèce (vivant librement, endoparasitoïde, ectoparasitoïde) est en corrélation avec sa propension à intégrer et à domestiquer des virus.

\section{Description d'une nouvelle famille de virus : les Filamentoviridae}

Depuis les années 1970, des études en microscopie électronique, ont mis en évidence une variété de virus dans le système reproducteur de guêpes parasitoïdes. En effet, plusieurs espèces de guêpes endoparasitoïdes présentaient des particules enveloppées en forme de bâtonnets flexibles à l'allure filamenteuse, ce qui leur a valu le nom de "virus filamenteux". Leur nature est néanmoins restée un mystère en raison du manque de données génétiques les concernant. De tels virus ont par exemple été décrits chez un Campopleginae (\textit{Diadegma terebrans} \citep{krell_replication_1987}), une série de  Braconidae (\textit{Cotesia congregata} \citep{de_buron_characterization_1992}, \textit{Cotesia hyphantriae}, \textit{Cotesia marginiventris} \citep{hamm_comparative_1990},  \textit{Microplitis rufiventris} \citep{hegazi_calyx_2005}) et un Figitidae (\textit{Leptopilina boulardi} \citep{varaldi_infectious_2003}).

Ce dernier virus que l'on retrouve chez \textit{Leptopilina boulardi} possède une biologie étonnante. Il s'agit d'un virus hérité verticalement de la mère aux progénitures se nommant Leptopilina boulardi filamentous virus (LbFV) en rapport à son hôte. Il manipule le comportement de ponte des femelles parasitoïdes en induisant du "superparasitisme" (ponte d'œufs dans des hôtes déjà parasités) \citep{varaldi_infectious_2003,varaldi_artifical_2006}. Cette modification du comportement favorise la transmission horizontale du virus au sein des larves de drosophile superparasitées. Cette modification du comportement est adaptative du point de vue du virus et coûteuse pour le parasitoïde \citep{gandon_superparasitism_2006}, ce qui fait de cette modification comportementale un exemple de "manipulation" induite par un virus.

C'est en 2017 que le premier "virus filamenteux", LbFV a été séquencé chez \textit{Leptopilina boulardi} \citep{lepetit_genome_2017}. Ce virus est par ailleurs apparenté au virus s'étant intégré chez les espèces du genre \textit{Leptopilina} et ayant mené à la domestication des 13 EVEs évoqués précédemment \citep{di_giovanni_behavior-manipulating_2020}. L'analyse phylogénétique l'a clairement identifié comme un membre de la classe des \textit{Naldaviricetes} (classe de virus à dsDNA infectant les arthropodes) dont la famille la plus proche est celle des \textit{Hytrosaviridae}. Néanmoins, en raison d'une divergence élevée avec les \textit{Hytrosaviridae}, il a été proposé que ce virus appartienne à une famille distincte au sein des \textit{Naldaviricetes}  \citep{lepetit_genome_2017}, dont il est le seul représentant.

Dans le \hyperref[sec:chap1]{chapitre 1}, pendant que je recherchais la présence systématique d'éléments viraux endogénisés dans les génomes d'Hyménoptères, nous avons également pu caractériser un certain nombre d'éléments viraux exogènes de virus ADN qui présentaient des signaux caractéristiques de virus libres. Ces virus à génome ADN correspondent à des virus ayant été séquencés au même moment que le génome de l'Hyménoptère. Nous avons ainsi pu découvrir chez 3 espèces de guêpes endoparasitoïdes, une vingtaine de scaffolds appartenant à 3 nouveaux virus libres proches phylogénétiquement de LbFV. Par ailleurs, nous avions aussi en notre possession le génome du virus LhFV retrouvé chez \textit{Leptopilina heterotoma} suite à une purification effectuée dans notre laboratoire.  Au même moment, l'équipe IRBI de Tours faisait des découvertes connexes chez une autre espèce de guêpes endoparasitoïdes où deux nouveaux virus ont été découverts.

C'est ainsi que le \hyperref[sec:chap1]{chapitre 2} est né d'une collaboration étroite avec l'IRBI de Tours, permettant de réunir des compétences et expériences variées. Dans cette collaboration, nous effectuons des analyses génomiques et phylogénétiques comparatives approfondies de ces six nouveaux virus filamenteux découverts chez cinq espèces d'Hyménoptères endoparasitoïdes.  Dans ce travail, des virus "filamenteux" infectant des guêpes appartenant à quatre super-familles et 5 genres : \textit{Leptopilina} (Cynipoidea), \textit{Encarsia} (Chalcidoidea), \textit{Platygaster} (Platygastroidea), \textit{Psyttalia}, et deux souches de \textit{Cotesia} (Ichneumonoidea). 

En utilisant des analyses phylogénétiques, génomique, et de morphogenèse des particules, nous montrons que les membres de cette nouvelle famille de virus filamenteux partagent quelques traits singuliers. Ces résultats apportent des arguments qui soutiennent l'hypothèse que les virus filamenteux constituent une famille de virus distincte au côté des \textit{Hytrosaviridae}. Ce travail constitue donc la base préliminaire à leur classification dans l'ICTV an tant que nouvelle famille que nous proposons de nommer Filamentoviridae en lien à leur structure filamenteuse, formant ainsi la cinquième famille de la classe des \textit{Naldaviricetes}.

\section{Caractérisation de l'évènement de domestication chez les \textit{Leptopilina} et estimation de l'âge des Filamentoviridae}.

Dans ce dernier travail de thèse (\hyperref[sec:chap1]{chapitre 3}), je me suis focalisé sur une échelle plus fine. Cette échelle est celle de la famille des Cynipoidea, une famille de guêpes endoparasitoïdes comprenant les \textit{Leptopilina}.  Nous venons de le voir, c'est précisément chez les espèces du genre \textit{Leptopilina} que se trouve le cas le plus récemment documenté de domestication de plusieurs gènes viraux (13). Ces gènes sont impliqués dans la synthèse des VLP, qui, lorsqu'elles sont injectées au côté des œufs, protègent la progéniture de la femelle du système immunitaire de l'hôte \citep{di_giovanni_behavior-manipulating_2020}. Ces 13 EVE domestiqués (dEVEs) sont issus d'un seul et même évènement d'intégration dont les gènes sont phylogénétiquement liés au virus filamenteux Leptopilina boulardi filamentous virus (LbFV) et proviennent donc d'un virus ancestral appartenant à la nouvelle famille des Filamentoviridae que nous décrivons dans le \hyperref[sec:chap1]{chapitre 2}. Avec ces nouveaux éléments, nous avions l'opportunité d'entrer dans le domaine de la paléovirologie. En effet, en étudiant ces virus anciens, figés dans les génomes des guêpes, nous pouvions maintenant décrire à la fois l'histoire évolutive de l'acquisition de ces gènes viraux (quand est-ce que l'endogénisation a eu lieu, et quelle est la diversité de guêpes impliquées ?), mais également en déduire la période d'apparition des virus de la famille des Filamentoviridae.

Pour mener ce projet, nous avons sondé les génomes de plusieurs espèces de la famille des Figitidae à laquelle appartiennent les \textit{Leptopilina}. Le jeu de donnée présentait 41 individus représentant 20 genres différents répartis en 6 sous-familles. 


%Les extractions d'ADN et l'amplification et le séquençage d'un produit PCR d'un ORF très conservé parmi les 13 nous ont permis de montrer que l'évènement concernait au moins les genres \textit{Leptopilina} et \textit{Trybliographa}, appartenant tous deux à la sous-famille des Eucoilinae (tribu des Eucoilini). Nous avons ensuite poursuivi l'analyse en échantillonnant, extrayant et séquençant le génome de 4 autres genres d'Eucoilinae (\textit{Rhoptromeris sp}, \textit{Leptolamina sp}, et \textit{Thrichoplasta sp}). Ces résultats nous ont permis d'inférer un évènement d'endogénisation ayant eu lieu après la divergence entre les \textit{Leptolamina} (espèce basale du groupe Eucoilini) et le reste du groupe, puisqu'aucune trace d'endogénisation n'a été détectée chez ce dernier tandis que l'ensemble des 13 gènes domestiqués ont été retrouvés dans les génomes des trois autres espèces. 

%Enfin, la caractérisation dans le temps de cet évènement d'endogénisation nous a permis de calibrer la phylogénie des virus libres s'étant intégré chez l'ancêtre commun de ces guêpes. En effet, puisque la phylogénie des 13 EVEs coïncide parfaitement avec celle des hôtes depuis l'ancêtre commun, la divergence des nœuds estimés chez les guêpes nous permet d'obtenir des points de calibration sur la phylogénie des virus \textit{Naldaviricetes} et ainsi d'estimer l'âge des différentes familles qui constituent ce clade. 

%Enfin, la caractérisation dans le temps de cet évènement d'endogénisation nous a permis de calibrer la phylogénie des virus apparentés au virus s'étant intégré il y a 75 millions d'années. En effet, puisque la phylogenèse des 13 EVEs coïncide parfaitement avec celle des hôtes depuis l'ancêtre commun, la divergence des nœuds estimés chez les Eucoilinae peut permettre d'obtenir des points de calibration sur la phylogénie des \textit{Naldaviricetes}. Ainsi, nos résultats coïncidents et précisent les estimations proposées en 2011 par \cite{theze_paleozoic_2011} pour la divergence des familles de \textit{Baculoviridae} et \textit{Nudiviridae}, il y a 348 millions d'années. Ces résultats permettent également de proposer une date de divergence de la classe des \textit{Naldaviricetes}, il y a 421 millions d'années, et celle de la famille des \textit{Filamentoviridae}, il y a 317 millions d'années. Aussi, c'est vraisemblablement à l'éon phanérozoïque, qui s'étend sur 538,8 millions d'années jusqu'à aujourd'hui, au cours duquel une vie animale et végétale abondante ont existé, que les grands virus à dsDNA de ce groupe auraient existé et auraient commencé à se diversifier et se spécialiser chez divers hôtes d'insectes.