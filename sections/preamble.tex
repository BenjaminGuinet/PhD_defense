\chapter*{Préambule}
\addcontentsline{toc}{part}{Préambule}
Imaginez-vous un instant à bord d’un vaisseau spatial de retour d’une mission commerciale de routine. Dix mois avant votre retour sur Terre, vous êtes tiré de votre léthargie par \textit{Mother} qui n’est autre que l’ordinateur de bord. \textit{Mother} détecte d’étranges signaux en provenance d’une planète inconnue : LV-426. Vous décidez de la visiter et y trouvez un ancien vaisseau abandonné contenant d’étranges œufs parfaitement alignés. L’un de vos coéquipiers, un peu trop curieux, effleure l’un de ces œufs et déclenche son éclosion. Une créature grêle et osseuse bondit et s’empresse de percer le scaphandre de l’homme, puis s’accroche à sa bouche avec vivacité. Vous venez de faire la rencontre des Facehuggers ! Aussitôt, vous reconduisez votre coéquipier au vaisseau pour décrocher cette créature de son visage, mais rien n’y fait. L’homme est plongé dans un profond coma. Quelques jours plus tard, vous assistez avec effroi à son réveil. Il agonise. Une étrange créature semble vouloir s’extirper de son corps. Après quelques secondes de cris abominables, une tête sombre et élancée surgit de la cage thoracique de votre ami. C’est à ce moment précis que vous comprenez que le signal préalablement détecté par \textit{Mother} n’était pas un appel à l’aide, mais une mise en garde. Il est maintenant trop tard, l’horreur est en marche…\\

Assez effrayant comme histoire n'est-ce-pas ? Les amateur.ices de films de frissons auront reconnu la célèbre saga de films Alien, tout droit sortie de l’imaginaire débordant de H.R. Giger et produite par Ridley Scott en 1979. Dans ces films, d’étranges créatures, appelées des Xénomorphes ou Aliens, mènent la vie dure aux astronautes. Et pour cause, il s’agit de créatures qui ont besoin d’autres organismes vivants pour compléter leur propre cycle de vie. Vous pourriez être apeuré par cette vision horrifique, et vous auriez sans doute raison de l’être. Mais que diriez-vous si je vous assurais que notre monde, bien réel, regorge de créatures au mode de vie similaire à ces Aliens ? C'est dans une lettre à l'intention d'Asa Gray le 22 mai 1860 que Charles Darwin fait part de son désarroi face à ces insectes particuliers : les parasitoïdes. \\

\textit{"En ce qui concerne le point de vue théologique de la question : cela m'est toujours douloureux. Je suis ahuri [...] Il me semble qu'il y a bien trop de misère dans ce monde. Je ne peux pas me persuader qu'un dieu bienfaisant et tout-puissant aurait créé les Ichneumonidae [(famille de guêpes parasitoïdes)] avec l'intention de les nourrir au sein des corps vivants de chenilles."}\\

Ces guêpes parasitoïdes, nous les avons tout.e.s croisées au moins une fois dans notre vie aux abords d'un champ ou près d'arbres fruitiers. C’est durant cette thèse que j’ai eu la chance de pouvoir les observer d’un peu plus près. Vous verrez que derrière leur biologie particulière se cachent de belles histoires qui illustrent à quel point notre monde regorge d'interactions étonnantes.