\thispagestyle{empty}

\section*{Résumé}
L'endogénisation accidentelle d'éléments viraux dans les génomes eucaryotes peut parfois apporter des avantages évolutifs importants, donnant lieu à leur domestication dans les génomes receveurs. Par exemple, chez certaines guêpes endoparasitoïdes (dont les stades immatures se développent à l'intérieur de leurs hôtes), la propriété de fusion membranaire des virus à ADN double brin a été domestiquée à plusieurs reprises suite à des endogénisations ancestrales. Les gènes endogénisés fournissent aux guêpes femelles un outil pour injecter des facteurs de virulence qui sont essentiels au succès du développement de leur progéniture. Dans ce contexte, mon travail de thèse représente une tentative de clarifier cette relation étroite entre le mode de vie endoparasitoïde et les phénomènes de domestication spécifiques de ces virus. Lors de mes travaux de thèse, nous avons testé l'hypothèse selon laquelle le mode de vie endoparasitoïde aurait favorisé l'endogénisation et la domestication des virus. Pour cela, nous avons analysé 124 génomes répartis dans la diversité des Hyménoptères comprenant des espèces libres et parasitoïdes. Nos analyses mettent en évidence que les virus à ADNdb sont effectivement plus fréquemment endogénisés et domestiqués et qu'ils le sont plus fréquemment dans les génomes de guêpes endoparasitoïdes. Ces résultats impliquent donc que l'endoparasitoïdisme est un facteur qui favorise les évènements de domestications, très probablement puisque les gènes viraux ainsi domestiqués leur permettent d'échapper à l'immunité de l'hôte. Puisque l'endogénisation d'un virus nécessite un contact entre ce virus et l'insecte, nous présumions retrouver des traces de virus encore infectieux chez plusieurs espèces de notre jeu de donnée. Ainsi, en analysant les données de séquençage, nous avons pu identifier plusieurs virus précédemment inconnus chez ces guêpes, y compris des proches parents d'un virus filamenteux précédemment décrit (LbFV) qui manipule le comportement de ponte des guêpes femelles qu'il infecte, favorisant ainsi sa propre transmission. Dans un travail collaboratif avec l'IRBI de Tours, nous avons décrit en détail ces nouveaux virus filamenteux, qui semblent préférentiellement liés au mode de vie des endoparasitoïdes. En outre, nous proposons qu'ils forment une nouvelle famille de virus, que nous recommandons de nommer Filamentoviridae. En combinant les résultats des deux recherches précédentes, nous présentons dans le troisième projet un exemple concret d'une interaction étroite entre un de ces virus filamenteux et une lignée de guêpes endoparasitoïdes de la tribu Eucoilini qui a conduit à la domestication de ce virus il y a 75 millions d'années. Autour de cette période, les hôtes de ces guêpes ont commencé à se diversifier, suggérant que la domestication de ces virus a eu un rôle important dans la capacité de ces guêpes à s'adapter à leurs hôtes. Ce travail nous a également permis de dater l'apparition des Filamentoviridae a environ 297 millions d'années, période d'apparition des premiers Hyménoptères. En conclusion, ces découvertes nous ont permis de mieux appréhender l'histoire évolutive des virus filamenteux et leurs liens privilégiés avec les guêpes endoparasitoïdes. Par ailleurs, l'implication des Filamentoviridae dans la biologie des Eucoilini soulève la question de leur rôle dans la biologie d'autres espèces d'endoparasitoïdes, que ce soit via la manipulation comportementale ou la domestication intra-génomique.\\


\section*{Abstract}
The accidental endogenisation of viral elements in eukaryotic genomes can sometimes provide important evolutionary advantages, leading to their long-term retention, i.e. viral domestication. For example, in some endoparasitoid wasps (whose immature stages develop inside their hosts), the membrane-fusion property of double-stranded DNA viruses has been domesticated repeatedly as a result of ancestral endogenisations. Endogenised genes provide female wasps with a tool to inject virulence factors that are essential for the successful development of their offspring. In this context, my thesis work represents an attempt to clarify the close relationship between endoparasitoids lifestyle and viruses. In my thesis work, we tested the hypothesis that the endoparasitoid lifestyle favoured the endogenisation and domestication of viruses. For this purpose, we analysed 124 Hymenoptera genomes distributed in the Hymenoptera diversity including free-living and parasitoid species. Our analyses show that dsDNA viruses are indeed more frequently endogenised and domesticated in the genomes of endoparasitoid wasps. These results imply that endoparasitoidism is a factor that favours domestication events, most likely because the domesticated viral genes allow them to escape host immunity. Since endogenisation of a virus requires contact between the virus and the insect, we assumed that traces of still-infectious virus genomes would be found in the genomic assemblies used in the first dataset. Indeed, by analysing the sequencing data, we were able to identify several previously unknown viruses in these wasps, including close relatives of a previously described filamentous virus (LbFV) that manipulates the egg-laying behaviour of the female wasps it infects, thereby promoting its own transmission. In a collaborative work with the IRBI of Tours, we have described in details these new filamentous viruses, which seem to be preferentially linked to the life style of endoparasitoids. Furthermore, we propose that they form a new viral family which we recommend to name Filamentoviridae. Combining the results of the two previous research projects, we present in the third project a concrete example of a close interaction between one of these filamentous viruses and a lineage of endoparasitoid wasps of the tribe Eucoilini that led to the domestication of this virus 75 million years ago. Around this time, the hosts of these wasps began to diversify, indicating that the domestication of these viruses probably had an important role in the ability of these wasps to adapt to their hosts. We also show that Filamentoviridae arose by the time of Hymenoptera emergence (around 297 Mya). In conclusion, these findings have allowed us to better understand the evolutionary history of dsDNA viruses and their privileged relationship with endoparasitoid wasps. Furthermore, the involvement of Filamentoviridae in the biology of Eucoilini raises the question of their role in the biology of other endoparasitoid species, either through behavioural manipulation or intra-genomic domestication.\\

\textbf{Key words:} HGT, EVE, domestication, dsDNA virus, endoparasitoïd.





%Puisque j'aime raconter des histoires, voici un petit résumé dont le but est surtout de présenter le sujet de mon travail sous un angle vulgarisé pour les personnes de mon entourage.

%L'histoire que je m'apprete à vous raconter commence il y a de ça très longtemps au moment même où le dinosaures arpentaient encore notre belle planète. A ce moment-là, dans une forêt lointaine, vivaient une population de guêpes parasitoïdes. Ces guêpes étaient uniques car, contrairement à leurs cousines, elles portaient dans leur corps un type particulier de matériel génétique viral appelé éléments viraux endogènes (EVE). Ces EVEs ont une histoire fascinante. Il y a longtemps, les ancêtres des guêpes avaient été exposés à un virus, et celui-ci avait inséré son matériel génétique dans l'ADN d'une guêpe par accident. Au fil du temps, ces gènes viraux sont devenus partie intégrante du patrimoine génétique des déscendants de cette guêpe et se sont transmis de génération en génération. Les EVE ont donné aux guêpes un avantage unique. Lorsqu'elles pondent leurs œufs dans le corps de leurs hôtes tels que des chenilles, les EVE confèrent à la progéniture des guêpes une immunité contre les défenses immunitaire de la chenille. Cela permettait aux oeufs des guêpes d'éclore et de se développer sans être détruits. La population de guêpes a prospéré grâce à cet avantage génétique, et elles sont vites devenues l'espèce dominante de la forêt. Les descendants des guêpes portent toujours ces EVEs dans leur corps et bénéficient toujours de l'immunité qu'elles procurent. Ces virus intégrés rappellent comment un seul virus, il y a plusieurs générations, a changé à jamais le cours de l'histoire de ces guêpes.

%C'est dans la lignée de cette histoire que mon travail de thèse s'inscrit. Durant ce travail de trois ans, j'ai eu la chance de découvrire d'autres aspects de cette histoire. J'ai par example pu apporter des évidences montrant que le style de vie endoparasitoïdes (c'est-à-dire que les guêpes pondent leurs oeufs à l'intérieur d'un hôte), était un facteur qui permettrait d'exliquer une plus grande probabilité d'intégrer et domestiquer du matériel génétique viral dont la motivation principale serait donc lié à une forte pression de séléction menée par les hôtes. 

%Dans un autre projet j'ai également pu découvrir une nouvelle famille de virus. Ces virus en forme de battonet s'appelent des virus filamenteux que nous avons choisit d'appeler des Filamentoviridae en lien avec leur structure singulière. Ce qui était intéréssant de découvrir, c'est que certains membres de cette famille s'étaient eux aussi intégré par accidant dans l'ADN de guêpes il y a de ça très longtemps chez l'ancêtre commun de plusieurs centaines d'espèces de guêpes de la tribbu des Eucoilini. Ces virus ont ensuite été domestiqué indépendemment de l'histoire précédente leur ont également permit d'acquérir cette fameuse abilité à défendre les oeufs des parasitoïde. 

%Mes derniers travaux m'ont permit d'étudier ce phénomène de domestication de virus chez les guêpes de la tribu des eucoilini de plus prêt. J'ai lors de se projet pu approcher le domaine de la paléovirologie en étudiant la nouvelle famille que nous venions de décrire.  La paléovirologie est l'étude des anciens virus qui sont encore vivants aujourd'hui sous la forme d'éléments viraux intégrés l'intérieur de l'ADN des guêpes (EVEs). Ces EVEs sont donc comme de minuscules capsules temporelles qui ont été conservées à l'intérieur des guêpes pendant des millions d'années. C'est donc en étudiant ces gènes figés dans le temps que j'ai pu étudier l'évolution des ancêtres des virus des Filmentoviridae. J'ai notamment pu proposer que ces virus vivaient il y a plus de 360 millions d'années au paléozoïque, période durant laquelle les hôtes des guêpes commençaient tout juste à se diversifier eux aussi.

%Au final, aventure qu'est la thèse est l'aboutissement de nombreuses années de travail et de dévouement. Je suis fier du travail que j'ai accompli et j'espère qu'il sera utile aux futurs chercheur.euse.s dans ce domaine. 



%Au cours de l'évolution, l'intégration accidentelle d'éléments viraux dans le génome eucaryote peut se produire. Ces éléments peuvent parfois conférer aux génomes récepteurs des avantages évolutifs majeurs : on parle alors de domestication virale. Par exemple, chez certaines guêpes endo-parasitoïdes (dont les stades immatures se développent à l'intérieur de leurs hôtes), la propriété fusogène des virus a été domestiquée à plusieurs reprises suite à des intégrations virales ancestrales d'ADNdb. Ces gènes viraux endogénisés sont utilisés par les guêpes femelles comme un outil de livraison pour injecter des facteurs de virulence, essentiels au succès du développement de leur progéniture. Comme tous les cas connus de domestication virale impliquent des guêpes endoparasites, nous avons émis l'hypothèse que ce mode de vie (favorisant une interaction étroite entre les individus) a pu favoriser l'endogénisation et la domestication des virus. En analysant la composition génomique de 124 génomes d'Hyménoptères répartis dans la diversité des Hyménoptères et incluant des espèces libres, ecto et endo-parasitoïdes, nous avons testé cette hypothèse. Notre analyse a d'abord révélé que les virus à ADN double brin étaient plus souvent endogénisés et domestiqués que prévu d'après leur abondance relative estimée dans les communautés virales infectant les insectes par rapport aux autres structures génomiques virales (ADNs, ARNdb, ARNsb). Deuxièmement, notre analyse a révélé que l'endogénisation et la domestication des virus à ADNdb étaient plus fréquentes dans les lignées ayant un mode de vie endo-parasitoïde par rapport aux lignées ayant un mode de vie ecto-parasitoïde ou libre. Ces résultats suggèrent donc que l'endoparasitoïsme a favorisé la domestication des virus à ADNdb ou inversement que la domestication des virus à ADNdb a favorisé l'évolution de l'endoparasitoïsme. Outre les bactéries symbiotiques, les insectes sont souvent associés à des virus héréditaires qui peuvent façonner leur phénotype. Cependant, la diversité de ces passagers clandestins est encore largement inconnue. Un exemple remarquable de leur contribution au phénotype des insectes concerne l'association entre le virus ADNdb LbFV et la guêpe endo-parasitoïde \textit{Leptopilina boulardi}.  Ces guêpes pondent leurs œufs dans des larves de drosophile, et le développement de leur progéniture conduit finalement à la mort de la drosophile.  Le virus héréditaire LbFV manipule le comportement de ponte de ces guêpes, en induisant un comportement de superparasitisme (ponte des œufs dans des hôtes déjà parasités). Cette modification du comportement favorise la transmission horizontale du virus au sein de l'hôte superparasité, permettant ainsi sa propagation efficace dans les populations de guêpes. Ce virus montre une proximité phylogénétique avec les \textit{Hytrosaviridae}, mais semble néanmoins appartenir à une nouvelle famille de virus. De manière intrigante, les analyses génomiques de plusieurs espèces de parasitoïdes (dont Leptopilina sp.) ont révélé la présence de multiples gènes d'origine virale provenant d'un ancien transfert horizontal massif impliquant un virus de type LbFV. Ces séquences virales ont été domestiquées par les guêpes parasites, car elles permettent la délivrance de facteurs immunosuppresseurs dans les cellules immunitaires de l'hôte, protégeant ainsi la progéniture des parasitoïdes de la réponse immunitaire de l'hôte. Dans cet exposé, je fournirai d'abord des preuves que le LbFV et quelques autres virus infectant des guêpes parasitoïdes apparentées lointaines constituent une nouvelle famille de virus à ADNdb. Je présenterai ensuite des données montrant que la plupart des espèces de parasitoïdes appartenant à la tribu Eucoilini (Cynipoidea, Figitidae) sont concernées par cet événement d'endogénisation, ce qui suggère que cet événement ne s'est produit qu'une seule fois il y a environ 75 millions d'années. 

