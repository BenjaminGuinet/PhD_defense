\thispagestyle{empty}
\chapter{Contexte général de la thèse}
{\hypersetup{linkcolor=GREYDARK}\minitoc}
\label{chap:intro-Context}
\newpage

Le cœur de ce projet s'inscrit dans la découverte antérieure de phénomènes de domestication virale chez plusieurs espèces de guêpes dont la progéniture se développe à l'intérieur du corps d'autres insectes; ces insectes étant qualifiés d'endoparasitoïdes. Ces virus sont utilisés aujourd’hui par les guêpes pour administrer des facteurs de virulence essentiels au succès développemental de leur progéniture. Au début de la thèse, cinq événements indépendants de domestication de virus avaient été documentés, dont quatre concernaient une seule super-famille de l’ordre des Hymenoptères (Ichneumonoidea), qui a focalisé l’attention des chercheurs.euses. Puisque ces événements s'étaient répétés à plusieurs reprises, la capacité de produire des particules virales devait conférer un avantage important sur le plan évolutif et jouer un rôle essentiel dans le succès écologique des espèces endoparasitoïdes. Néanmoins, à ce stade, nous manquions d’une vision globale de la fréquence de ces phénomènes de domestication de virus au sein de l’immense diversité des Hyménoptères parasitoïdes. En effet, jusqu'alors, par l'implication de ces espèces dans la lutte contre les ravageurs de cultures, seulement quelques espèces de la super-famille des Ichneumonoidea avaient été séquencées et étudiées, ce qui soulevait clairement un biais dans l'effort d'échantillonnage au sein de la diversité des Hyménoptères. Par exemple, la super-famille des Chalcidoidea est l'une des plus diversifiées au sein des Hyménoptères, en concurrence avec les Ichneumonoidea, avec plus de 22 500 espèces existantes décrites et une estimation allant jusqu'à 500 000 espèces \citep{heraty_phylogenetic_2013}. \\

Le présent projet de thèse vise à rechercher des facteurs qui pourraient structurer les évènements d'endogenisation et de domestication à l'échelle des Hyménoptères. Pour cela, nous analysons un maximum de diversité et incluons des facteurs biologiques différents (espèces endoparasitoïdes, ectoparasitoïdes, et libres) au sein des Hyménoptères, et ceci en utilisant des génomes présents dans les bases de données publiques et séquencées par notre laboratoire.  J'ai  donc développé un pipeline bio-informatique permettant de rechercher de nouveaux cas d'endogénisation et de domestication, et par ailleurs de tester l'existence de facteurs explicatifs  associés à l'apparition de ces phénomènes.\\

En travaillant sur mon projet principal, j'ai fait des découvertes inattendues, ce qui est un événement courant en science. Lors du criblage d'éléments viraux endogènes dans les assemblages de centaines d'espèces d'Hyménoptères, nous avons découvert la présence de virus libres non identifiés chez certaines. Au même moment, une équipe de l'IRBI (Institut de Recherche sur la Biologie de l'Insecte) de Tours faisait également des découvertes connexes chez des virus filamenteux apparentés. Ces travaux ont donc conduit à un projet commun entre nos deux équipes et m'ont offert l'opportunité de décrire une nouvelle famille de virus, tout en apprenant de nouveaux concepts.\\

Enfin, j'ai pu combiner nombre d'idées et de résultats issus de mes deux projets précédents pour le dernier projet de cette thèse. L'objectif de celui-ci était de fournir un compte rendu détaillé de la domestication virale qui s'est produite chez les guêpes endoparasitoïdes du genre \textit{Leptopilina}. C'est lors de ce projet que j'ai pu mettre un pied dans le domaine de la biologie moléculaire en effectuant l'extraction d'ADN et la recherche par PCR d'éléments viraux endogènes sur une quarantaine d'échantillons provenant d'espèces proches des \textit{Leptopilina}. L'idée était de cribler un maximum de diversité autour des espèces de \textit{Leptopilina} afin de remonter jusqu'à l'évènement ancestral d'endogénisation.
Ces analyses m'ont notamment permis de préciser la date à laquelle ces gènes d'origine virale ont été acquis par ces guêpes, puis d'inférer la longue histoire évolutive de la nouvelle famille de virus décrite précédemment. \\

Ce travail de thèse est le fruit de nombreuses collaborations, qu'elles soient techniques via le séquençage de nombreux génomes, participatives grâce au partage de nombreux spécimens, développementales à travers la conception de scripts bioinformatiques, ou bien intellectuelles par les échanges au cours de nombreuses discussions. Aussi, ce voyage n’aurait jamais été si passionnant sans l'aide des nombreuses personnes impliquées de près ou de loin dans ma thèse.\\

Cette thèse est présentée en vue de satisfaire aux exigences du diplôme de Docteur à l'Université de Lyon. Les recherches présentées ici ont été conduites au Laboratoire de Biométrie et de Biologie Évolutive (LBBE), sous la supervision du Maître de conférence, Dr. Julien Varaldi. La thèse est un recueil de trois manuscrits précédés d'une introduction qui les met en relation et fournit des informations générales, suivie finalement d'une discussion générale et d'une mise en perspective de mon travail.\\

Ce travail a été réalisé en utilisant les ressources informatiques du CC LBBE/PRABI ainsi que de l'IFB Biosphère. \\

\newline


\newcolumntype{C}[1]{>{\centering\let\newline\\\arraybackslash\hspace{0pt}}m{#1}}
\newcolumntype{L}[1]{>{\raggedright\let\newline\\\arraybackslash\hspace{0pt}}m{#1}}

    \begin{tabular}{C{2.8cm}  L{5.5cm}}
        \includegraphics[width=\linewidth]{PhD-master/figures/GitHub-logo.png} & Toutes les figures, les scripts et le code source LATEX utilisés dans ce manuscrit peuvent être réutilisés sous licence CC-BY-SA, disponible à l'adresse : \newline
        \href{https://github.com/BenjaminGuinet/PhD_defense}{https://github.com/BenjaminGuinet/PhD\_defense}
    \end{tabular}
